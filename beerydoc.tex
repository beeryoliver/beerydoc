\documentclass{beerydoc}


\begin{document}


\section*{\stepfontsize{1}The \cls{beerydoc} class}

Class designed for the documentation of my packages

Oliver Beery

Version 0.0.0\quad 19 December 2025


\section{About}

Documentation for the \cls{beerydoc} class.

\subsection{Loading the class}

\listheading{Requirements:}
\begin{itemize}
  \item \LaTeXe{} version 2023-11-01 or newer
  \item \pkg{l3kernel} version 2023-11-01 or newer
  \item Must be compiled with Lua\TeX.
\end{itemize}

\subsection{Package option}

\begin{variable}{showframe}
  \begin{syntax}
    \key{showframe}
  \end{syntax}
  Passes option \key{showframe} to the \pkg{geometry} package.
\end{variable}

\section{Commands}

\subsection{Markup for package and class commands}

\begin{function}{\key}
  \begin{syntax}
    \cs{key} \marg*{\meta{key}=\meta{value}}
    \cs{key} \marg{key}
    \cs{key} \marg*{=\meta{value}}
  \end{syntax}
  Prints a key-value pair, a key, or the value of a key.
  The leading |=| in \marg*{=\meta{value}} is omitted.
  \meta{key} and \meta{value} print their contents in \cs{ttfamily}.
  The |=| in \marg*{\meta{key}=\meta{value}} is reformatted and a line break
  is allowed afterwards.
  Within \meta{key}, line breaks are allowed after each |/|.
  Within \meta{value}, spaces are replaced by \cs{textvisiblespace}.
\end{function}

\begin{function}{\choices}
  \begin{syntax}
    \cs{choices} \marg{comma list}
  \end{syntax}
  This command is defined only in \meta{value} in \cs{key}.
  Prints the available choices in \meta{comma list} when documenting a choice
  key.
\end{function}

\begin{function}{\cs}
  \begin{syntax}
    \cs{cs} \marg{command}
  \end{syntax}
  Prints a command.
  \meta{command} must omit the leading backslash |\|.
  The contents are printed in \cs{ttfamily}.
  \cs{meta} can be used in \meta{command}.
  The underscore |_| can be typeset directly when documenting \pkg{expl3}
  commands.
  Spaces are replaced by \cs{textvisiblespace}.
  % A double at symbol |@@| will be replaced by the module name using the same
  % convention used in \pkg{docstrip}.
  Active vertical bars \texttt{\textbar} are replaced by that of catcode 12.
\end{function}

\begin{function}{\meta}
  \begin{syntax}
    \cs{meta} \marg{meta}
  \end{syntax}
  Prints a meta representation such as \meta{token list}, \meta{string}, and
  \meta{comma list}.
  The contents are printed in \cs{ttfamily} and \cs{slshape}.
  The underscore |_| is used to apply a subscript, e.g.\ \meta{token list_1},
  \meta{string_2}, \meta{comma list_3}.

  \listheading
    {
      If \meta{meta} begins with the tokens on the left, then they will be
      replaced by the tokens on the right:
    }
  \begin{itemize}[parsep=0pt]
    \item |fp expr| \rightarrow{} |floating point expression|
    \item |int expr| \rightarrow{} |integer expression|
    \item |dim expr| \rightarrow{} |dimen expression|
    \item |skip expr| \rightarrow{} |skip expression|
    \item |muskip expr| \rightarrow{} |muskip expression|
  \end{itemize}
\end{function}

\begin{function}{\marg}
  \begin{syntax}
    \cs{marg} \sarg{} \marg{mandatory argument}
  \end{syntax}
  Prints a mandatory argument that is enclosed in braces.
  The contents are printed in \cs{ttfamily}.
  If the optional star is omitted, \meta{mandatory argument} is formatted
  using \cs{meta}\marg{mandatory argument}.
\end{function}

\begin{function}{\oarg}
  \begin{syntax}
    \cs{oarg} \sarg{} \marg{optional argument}
  \end{syntax}
  Prints an optional argument that is enclosed in square brackets.
  The contents are printed in \cs{ttfamily}.
  If the optional star is omitted, \meta{optional argument} is formatted using
  \cs{meta}\marg{optional argument}.
\end{function}

\begin{function}{\sarg}
  \begin{syntax}
    \cs{sarg}
  \end{syntax}
  Prints an optional star that is enclosed in angle brackets.
  The optional star is printed in \cs{ttfamily}.
\end{function}

\begin{function}{\cls, \env, \pkg, \hook, \socket, \plug, \template, \instance}
  \begin{syntax}
    \obeyspaces
    \cs{cls}      \marg{class}
    \cs{pkg}      \marg{package}
    \cs{env}      \marg{environment}
    \cs{hook}     \marg{hook}
    \cs{socket}   \marg{socket}
    \cs{plug}     \marg{plug}
    \cs{template} \marg{template}
    \cs{instance} \marg{instance}
  \end{syntax}
  Prints the name of a class, environment, package, hook, socket, plug,
  template, or instance.
  The contents are printed in \cs{ttfamily}, except for \meta{class} and
  \meta{package} which are printed in \cs{sffamily}.
\end{function}

\subsection{Some text formatting}

\begin{function}{\allcaps}
  \begin{syntax}
    \cs{allcaps} \oarg{int expr} \marg{text}
  \end{syntax}
  Prints \meta{text} in all capital letters with added letterspacing.
  The amount of letterspacing equals the result of evaluating the
  \meta{int expr} divided by \num{1000} in units of \unit{em}.
  If the optional argument is omitted, the letterspacing is increased by
  \qty{0.05}{em}.
\end{function}

\begin{function}{\allsmallcaps}
  \begin{syntax}
    \cs{allsmallcaps} \sarg{} \oarg{int expr} \marg{text}
  \end{syntax}
  Prints \meta{text} in all small caps with added letterspacing.
  The amount of letterspacing equals the result of evaluating the
  \meta{int expr} divided by \num{1000} in units of \unit{em}.
  If the optional argument is omitted, the letterspacing is increased by
  \qty{0.05}{em}.
  If the optional star is added, capital letters are also changed to small
  caps.
\end{function}

\begin{function}{\honorific}
  \begin{syntax}
    \cs{honorific} \marg{honorific}
  \end{syntax}
  Prints an honorific such as Mr., Ms., or Dr.
  This command should only be used before the name of the person.
\end{function}

\begin{function}{\latin}
  \begin{syntax}
    \cs{latin} \marg{text}
  \end{syntax}
  Prints a latin phrase.
  The contents are printed in \cs{itshape}.
\end{function}

\subsection{Expandable commands}

Some expandable commands that are useful for testing.

\begin{function}{\alphabet}
  \begin{syntax}
    \cs{alphabet}
  \end{syntax}
  Expands to \enquote{\alphabet}.
\end{function}

\begin{function}{\digits}
  \begin{syntax}
    \cs{digits}
  \end{syntax}
  Expands to \enquote{\digits}.
\end{function}

\begin{function}{\pangram}
  \begin{syntax}
    \cs{pangram}
  \end{syntax}
  Expands to \enquote{\pangram}
\end{function}


\section{Environments}

\begin{function}{function, variable}
  \begin{syntax}
    \cs{begin}\marg*{function}\marg{comma list}
    \cs{begin}\marg*{variable}\marg{comma list}
  \end{syntax}
  Used to document one or more functions or variables.
  \meta{comma list} should contain a comma-separated list of the documented
  functions or variables.
  The content is contained within a \env{minipage} to prevent page breaks.
\end{function}

\begin{function}{syntax}
  \begin{syntax}
    \cs{begin}\marg*{syntax}
  \end{syntax}
  Used to document the syntax of a function (and sometimes variable) with a
  \env{function} or \env{variable} environment.
  The contents are printed in \cs{ttfamily}.
  \cs{obeylines} is enabled within a \env{syntax} environment.
\end{function}


\end{document}
